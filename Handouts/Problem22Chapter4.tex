\documentclass[ignorenonframetext,]{beamer}
\usetheme{Madrid}
\usepackage{amssymb,amsmath}
\usepackage{ifxetex,ifluatex}
\usepackage{fixltx2e} % provides \textsubscript
\ifxetex
  \usepackage{fontspec,xltxtra,xunicode}
  \defaultfontfeatures{Mapping=tex-text,Scale=MatchLowercase}
\else
  \ifluatex
    \usepackage{fontspec}
    \defaultfontfeatures{Mapping=tex-text,Scale=MatchLowercase}
  \else
    \usepackage[utf8]{inputenc}
  \fi
\fi
\usepackage{color}
\usepackage{fancyvrb}
\newcommand{\VerbBar}{|}
\newcommand{\VERB}{\Verb[commandchars=\\\{\}]}
\DefineVerbatimEnvironment{Highlighting}{Verbatim}{commandchars=\\\{\}}
% Add ',fontsize=\small' for more characters per line
\usepackage{framed}
\definecolor{shadecolor}{RGB}{248,248,248}
\newenvironment{Shaded}{\begin{snugshade}}{\end{snugshade}}
\newcommand{\KeywordTok}[1]{\textcolor[rgb]{0.13,0.29,0.53}{\textbf{{#1}}}}
\newcommand{\DataTypeTok}[1]{\textcolor[rgb]{0.13,0.29,0.53}{{#1}}}
\newcommand{\DecValTok}[1]{\textcolor[rgb]{0.00,0.00,0.81}{{#1}}}
\newcommand{\BaseNTok}[1]{\textcolor[rgb]{0.00,0.00,0.81}{{#1}}}
\newcommand{\FloatTok}[1]{\textcolor[rgb]{0.00,0.00,0.81}{{#1}}}
\newcommand{\CharTok}[1]{\textcolor[rgb]{0.31,0.60,0.02}{{#1}}}
\newcommand{\StringTok}[1]{\textcolor[rgb]{0.31,0.60,0.02}{{#1}}}
\newcommand{\CommentTok}[1]{\textcolor[rgb]{0.56,0.35,0.01}{\textit{{#1}}}}
\newcommand{\OtherTok}[1]{\textcolor[rgb]{0.56,0.35,0.01}{{#1}}}
\newcommand{\AlertTok}[1]{\textcolor[rgb]{0.94,0.16,0.16}{{#1}}}
\newcommand{\FunctionTok}[1]{\textcolor[rgb]{0.00,0.00,0.00}{{#1}}}
\newcommand{\RegionMarkerTok}[1]{{#1}}
\newcommand{\ErrorTok}[1]{\textbf{{#1}}}
\newcommand{\NormalTok}[1]{{#1}}
\usepackage{graphicx}
% Redefine \includegraphics so that, unless explicit options are
% given, the image width will not exceed the width of the page.
% Images get their normal width if they fit onto the page, but
% are scaled down if they would overflow the margins.
\makeatletter
\def\ScaleIfNeeded{%
  \ifdim\Gin@nat@width>\linewidth
    \linewidth
  \else
    \Gin@nat@width
  \fi
}
\makeatother
\let\Oldincludegraphics\includegraphics
\renewcommand{\includegraphics}[2][]{\Oldincludegraphics[width=\ScaleIfNeeded]{#2}}

% Comment these out if you don't want a slide with just the
% part/section/subsection/subsubsection title:
\AtBeginPart{
  \let\insertpartnumber\relax
  \let\partname\relax
  \frame{\partpage}
}
\AtBeginSection{
  \let\insertsectionnumber\relax
  \let\sectionname\relax
  \frame{\sectionpage}
}
\AtBeginSubsection{
  \let\insertsubsectionnumber\relax
  \let\subsectionname\relax
  \frame{\subsectionpage}
}

\setlength{\parindent}{0pt}
\setlength{\parskip}{6pt plus 2pt minus 1pt}
\setlength{\emergencystretch}{3em}  % prevent overfull lines
\setcounter{secnumdepth}{0}

\title{Chapter 4 Problem 22}
\author{Alan Arnholt}
\date{March 27, 2014}

\begin{document}
\frame{\titlepage}

\begin{frame}{Problem 22}

Let $X_1, X_2, \ldots X_n \overset{iid}\sim F$ with corresponding pdf
$f(x) = 3x^2, 0\leq x \leq 1.$

\begin{enumerate}[(a)]
\item Find the pdf for $X_{\textrm{min}}.$ \label{parta}
\pause
\item Find the pdf for $X_{\textrm{max}}.$ \label{partb}
\pause
\item If $n = 10,$ find the probability that the largest value, $X_{\textrm{max}},$ is greater than 0.92 \label{partc}
\pause
\item If $n = 10,$ find the expected value of $X_{\textrm{max}}.$ \label{partd}
\end{enumerate}

\end{frame}

\begin{frame}{$f_{\textrm{min}}(x)$}

\eqref{parta} The pdf for $X_{\textrm{min}}$ is:

\begin{equation}\label{Xmin}
f_{\textrm{min}}(x) = n\left(1 - F(X)\right)^{n-1}f(x)
\end{equation}

Since $F(x) = \int_{0}^{x}3t^2\,dt = x^3, 0\leq x \leq 1,$ it follows
that
\[f_{\textrm{min}}(x) = n\left(1 - x^3\right)^{n-1}3x^2,  0\leq x \leq 1\]

\end{frame}

\begin{frame}{$f_{\textrm{max}}(x)$}

\eqref{partb} The pdf for $X_{\textrm{max}}$ is:

\begin{equation}\label{Xmax}
f_{\textrm{max}}(x) = nF^{n-1}(x)f(x)
\end{equation}

Since $F(x) = \int_{0}^{x}3t^2\,dt = x^3, 0\leq x \leq 1,$ it follows
that
\[f_{\textrm{max}}(x) = n\left(x^3\right)^{n-1}3x^2,  0\leq x \leq 1\]

\end{frame}

\begin{frame}[fragile]{$P(X_{\textrm{max}} > 0.92)$}

\eqref{partc} Using \eqref{Xmax} gives:
\[f_{\textrm{max}}(x) = 10\left(x^3\right)^{10-1}3x^2 = 30x^{29},  0\leq x \leq 1\]

\[P(X_{\textrm{max}} > 0.92) = \int_{0.92}^{1}30x^{29}\,dx\]

\begin{Shaded}
\begin{Highlighting}[]
\NormalTok{f <-}\StringTok{ }\NormalTok{function(x)\{}\DecValTok{30}\NormalTok{*x^}\DecValTok{29}\NormalTok{\}}
\NormalTok{ans <-}\StringTok{ }\KeywordTok{integrate}\NormalTok{(f, }\FloatTok{0.92}\NormalTok{, }\DecValTok{1}\NormalTok{)$value}
\NormalTok{ans}
\end{Highlighting}
\end{Shaded}

\begin{verbatim}
[1] 0.918
\end{verbatim}

\[P(X_{\textrm{max}} > 0.92) = \int_{0.92}^{1}30x^{29}\,dx = 0.918.\]

\end{frame}

\begin{frame}[fragile]{$E[X_{\textrm{max}}]$}

\eqref{partd}
\[E[X_{\textrm{max}}] = \int_{0}^{1}x_{\textrm{max}} f_{\textrm{max}}(x) = \int_{0}^{1}30x^{30}\]

\begin{Shaded}
\begin{Highlighting}[]
\NormalTok{f1 <-}\StringTok{ }\NormalTok{function(x)\{}\DecValTok{30}\NormalTok{*x^}\DecValTok{30}\NormalTok{\}}
\NormalTok{ans <-}\StringTok{ }\KeywordTok{integrate}\NormalTok{(f1, }\DecValTok{0}\NormalTok{, }\DecValTok{1}\NormalTok{)$value}
\NormalTok{ans}
\end{Highlighting}
\end{Shaded}

\begin{verbatim}
[1] 0.9677
\end{verbatim}

\[E[X_{\textrm{max}}] = 0.9677\]

\end{frame}

\begin{frame}[fragile]{What does the pdf look like?}

\begin{Shaded}
\begin{Highlighting}[]
\KeywordTok{curve}\NormalTok{(f, }\FloatTok{0.85}\NormalTok{, }\DecValTok{1}\NormalTok{, }\DataTypeTok{n =} \DecValTok{500}\NormalTok{, }\DataTypeTok{col =}\StringTok{"purple"}\NormalTok{, }
      \DataTypeTok{xlab =} \KeywordTok{expression}\NormalTok{(x[max]), }
      \DataTypeTok{ylab =} \KeywordTok{expression}\NormalTok{(f[max](x)))}
\KeywordTok{abline}\NormalTok{(}\DataTypeTok{v =} \NormalTok{ans, }\DataTypeTok{lty =} \StringTok{"dashed"}\NormalTok{, }\DataTypeTok{col =} \StringTok{"red"}\NormalTok{)}
\end{Highlighting}
\end{Shaded}

\includegraphics{Problem22Chapter4_files/figure-beamer/unnamed-chunk-2.pdf}

\end{frame}

\begin{frame}[fragile]{What does the pdf look like with
\texttt{ggplot2}?}

\begin{Shaded}
\begin{Highlighting}[]
\KeywordTok{library}\NormalTok{(ggplot2)}
\KeywordTok{ggplot}\NormalTok{(}\DataTypeTok{data =} \KeywordTok{data.frame}\NormalTok{(}\DataTypeTok{x =} \KeywordTok{c}\NormalTok{(}\FloatTok{0.85}\NormalTok{, }\DecValTok{1}\NormalTok{)), }\KeywordTok{aes}\NormalTok{(}\DataTypeTok{x =} \NormalTok{x)) +}\StringTok{ }
\StringTok{  }\KeywordTok{stat_function}\NormalTok{(}\DataTypeTok{fun =} \NormalTok{f, }\DataTypeTok{color =} \StringTok{"purple"}\NormalTok{) +}\StringTok{ }
\StringTok{  }\KeywordTok{theme_bw}\NormalTok{() +}
\StringTok{  }\KeywordTok{geom_vline}\NormalTok{(}\DataTypeTok{xinter =} \NormalTok{ans, }\DataTypeTok{lty =} \StringTok{"dashed"}\NormalTok{, }\DataTypeTok{color =} \StringTok{"red"}\NormalTok{) +}\StringTok{ }
\StringTok{  }\KeywordTok{labs}\NormalTok{(}\DataTypeTok{x =} \KeywordTok{expression}\NormalTok{(x[max]), }\DataTypeTok{y =} \KeywordTok{expression}\NormalTok{(f[max](x)))}
\end{Highlighting}
\end{Shaded}

\includegraphics{Problem22Chapter4_files/figure-beamer/unnamed-chunk-3.pdf}

\end{frame}

\end{document}
